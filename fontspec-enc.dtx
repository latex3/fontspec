
% \section{Extended font encodings}
%
% \iffalse
%    \begin{macrocode}
%<*fontspec&(xetexx|luatex)>
%    \end{macrocode}
% \fi

%    \begin{macrocode}

\providecommand\UnicodeFontFile[2]{"[#1]:#2"}
\providecommand\UnicodeFontName[2]{"#1:#2"}

\providecommand\add@unicode@accent[2]{#2\char#1\relax}
\providecommand\DeclareUnicodeAccent[3]{%
  \DeclareTextCommand{#1}{#2}{\add@unicode@accent{#3}}%
}

%    \end{macrocode}
%
%    \begin{macrocode}

\DeclareDocumentCommand\EncodingCommand{mO{}m}{%
  \bool_if:NF \l_@@_defining_encoding_bool { \ERROR }
  \DeclareTextCommand{#1}{\LastDeclaredEncoding}[#2]{#3}%
}

\def\EncodingAccent#1#2{%
  \bool_if:NF \l_@@_defining_encoding_bool { \ERROR }
  \DeclareTextCommand{#1}{\LastDeclaredEncoding}{\add@unicode@accent{#2}}%
}

\def\EncodingSymbol#1#2{%
  \bool_if:NF \l_@@_defining_encoding_bool { \ERROR }
  \DeclareTextSymbol{#1}{\LastDeclaredEncoding}{#2}%
}

\def\EncodingComposite#1#2#3{%
  \bool_if:NF \l_@@_defining_encoding_bool { \ERROR }
  \DeclareTextComposite{#1}{\LastDeclaredEncoding}{#2}{#3}%
}

\def\EncodingCompositeCommand#1#2#3{%
  \bool_if:NF \l_@@_defining_encoding_bool { \ERROR }
  \DeclareTextCompositeCommand{#1}{\LastDeclaredEncoding}{#2}{#3}%
}


%% COMMANDS FOR DEFINING NEW ENCODINGS FROM FONT RANGES

\cs_new:Nn \@@_new_unicode_encoding:n {%
  \DeclareFontEncoding{#1}{}{}
  \DeclareErrorFont{\LastDeclaredEncoding}{lmr}{m}{n}{10}
  \DeclareFontSubstitution{\LastDeclaredEncoding}{lmr}{m}{n}
  \DeclareFontFamily{\LastDeclaredEncoding}{lmr}{}
  \DeclareFontShape{\LastDeclaredEncoding}{lmr}{m}{n}
       {<->\UnicodeFontFile{lmroman10-regular}{\UnicodeFontTeXLigatures}}{}
  \DeclareFontShape{\LastDeclaredEncoding}{lmr}{m}{it}
       {<->\UnicodeFontFile{lmroman10-italic}{\UnicodeFontTeXLigatures}}{}
  \DeclareFontShape{\LastDeclaredEncoding}{lmr}{m}{sc}
       {<->\UnicodeFontFile{lmromancaps10-regular}{\UnicodeFontTeXLigatures}}{}
  \DeclareFontShape{\LastDeclaredEncoding}{lmr}{bx}{n}
       {<->\UnicodeFontFile{lmroman10-bold}{\UnicodeFontTeXLigatures}}{}
  \DeclareFontShape{\LastDeclaredEncoding}{lmr}{bx}{it}
       {<->\UnicodeFontFile{lmroman10-bolditalic}{\UnicodeFontTeXLigatures}}{}
}

\DeclareDocumentCommand \DeclareUnicodeEncoding {mm} {
  \@@_new_unicode_encoding:n {#1}
  \bool_set_true:N \l_@@_defining_encoding_bool
  #2
  \bool_set_false:N \l_@@_defining_encoding_bool
}

\DeclareDocumentCommand \ImportTU {} {
  \bool_if:NF \l_@@_defining_encoding_bool { \ERROR }
  \tl_set_eq:NN \l_@@_unicode_name_tl \UnicodeEncodingName
  \tl_set_eq:NN \UnicodeEncodingName \LastDeclaredEncoding
  \ProvidesFile{tuenc.def}
    [2015/12/31 v0.1 Unicode font encoding for LaTeX2e]

\DeclareFontEncoding{TU}{}{}
\DeclareErrorFont{TU}{lmr}{m}{n}{10}
\DeclareFontSubstitution{TU}{lmr}{m}{n}

%% FONT LOADING (.fd FILE COMMANDS)

\begingroup\expandafter\expandafter\expandafter\endgroup
\expandafter\ifx\csname XeTeXrevision\endcsname\relax\else
  \def\UnicodeFontTeXLigatures{mapping=tex-text;}
\fi

\begingroup\expandafter\expandafter\expandafter\endgroup
\expandafter\ifx\csname directlua\endcsname\relax\else
  \def\UnicodeFontTeXLigatures{+tlig;+trep;}
\fi

\def\UnicodeFontFile#1#2{"[#1]:#2"}
\def\UnicodeFontName#1#2{"#1:#2"}

%% COMMANDS FOR SYMBOL DEFINITIONS (used below)

% Accents in Unicode are postpended: 
\def\add@unicode@accent#1#2{#2\char#1\relax}

% Replacement for \DeclareTextAccent:
\def\DeclareUnicodeAccent#1#2{%
  \DeclareTextCommand{#1}{TU}{\add@unicode@accent{#2}}%
}

%%%%%%%%%%%%%%%%%%%%%%%%%%%%%%%%%%%
%% T1 SYMBOLS

\DeclareUnicodeAccent{\`}{"0300}
\DeclareUnicodeAccent{\'}{"0301}
\DeclareUnicodeAccent{\^}{"0302}
\DeclareUnicodeAccent{\~}{"0303}
\DeclareUnicodeAccent{\"}{"0308}
\DeclareUnicodeAccent{\H}{"030B}
\DeclareUnicodeAccent{\r}{"030A}
\DeclareUnicodeAccent{\v}{"030C}
\DeclareUnicodeAccent{\u}{"0306}
\DeclareUnicodeAccent{\=}{"0304}
\DeclareUnicodeAccent{\.}{"0307}
\DeclareUnicodeAccent{\b}{"0332}
\DeclareUnicodeAccent{\c}{"0327}
\DeclareUnicodeAccent{\d}{"0323}
\DeclareUnicodeAccent{\k}{"0328}
%% \textogonekcentered %% not in unicode?
\DeclareTextSymbol{\textperthousand}{TU}{"2030}
\DeclareTextSymbol{\textpertenthousand}{TU}{"2031}
\DeclareTextSymbol{\AE}{TU}{"00C6}
\DeclareTextSymbol{\DH}{TU}{"00D0}
\DeclareTextSymbol{\DJ}{TU}{"0110}
\DeclareTextSymbol{\L} {TU}{"0141}
\DeclareTextSymbol{\NG}{TU}{"014A}
\DeclareTextSymbol{\OE}{TU}{"0152}
\DeclareTextSymbol{\O} {TU}{"00D8}
\DeclareTextSymbol{\SS}{TU}{"1E9E}
\DeclareTextSymbol{\TH}{TU}{"00DE}
\DeclareTextSymbol{\ae}{TU}{"00E6}
\DeclareTextSymbol{\dh}{TU}{"00F0}
\DeclareTextSymbol{\dj}{TU}{"0111}
\DeclareTextSymbol{\guillemotleft}{TU}{"00AB}
\DeclareTextSymbol{\guillemotright}{TU}{"00BB}
\DeclareTextSymbol{\guilsinglleft}{TU}{"2039}
\DeclareTextSymbol{\guilsinglright}{TU}{"203A}
\DeclareTextSymbol{\i} {TU}{"0131}
\DeclareTextSymbol{\j} {TU}{"0237}
\DeclareTextSymbol{\ij}{TU}{"0133}
\DeclareTextSymbol{\IJ}{TU}{"0132}
\DeclareTextSymbol{\l} {TU}{"0142}
\DeclareTextSymbol{\ng}{TU}{"014B}
\DeclareTextSymbol{\oe}{TU}{"0153}
\DeclareTextSymbol{\o} {TU}{"00F8}
\DeclareTextSymbol{\quotedblbase}{TU}{"201E}
\DeclareTextSymbol{\quotesinglbase}{TU}{"201A}
\DeclareTextSymbol{\ss}{TU}{"00DF}
\DeclareTextSymbol{\textasciicircum}{TU}{`\^}
\DeclareTextSymbol{\textasciitilde}{TU}{`\~}
\DeclareTextSymbol{\textbackslash}{TU}{`\\}
\DeclareTextSymbol{\textbar}{TU}{`\|}
\DeclareTextSymbol{\textbraceleft}{TU}{`\{}
\DeclareTextSymbol{\textbraceright}{TU}{`\}}
%% \DeclareTextSymbol{\textcompwordmark}{TU}{23}
\DeclareTextSymbol{\textdollar}{TU}{`\$}
\DeclareTextSymbol{\textemdash}{TU}{"2014}
\DeclareTextSymbol{\textendash}{TU}{"2013}
\DeclareTextSymbol{\textexclamdown}{TU}{"00A1}
\DeclareTextSymbol{\textgreater}{TU}{`\>}
\DeclareTextSymbol{\textless}{TU}{`\<}
\DeclareTextSymbol{\textquestiondown}{TU}{"00BF}
\DeclareTextSymbol{\textquotedblleft}{TU}{"201C}
\DeclareTextSymbol{\textquotedblright}{TU}{"201D}
\DeclareTextSymbol{\textquotedbl}{TU}{`\"}
\DeclareTextSymbol{\textquoteleft}{TU}{`\`}
\DeclareTextSymbol{\textquoteright}{TU}{`\'}
\DeclareTextSymbol{\textsection}{TU}{"00A7}
\DeclareTextSymbol{\textsterling}{TU}{"00A3}
\DeclareTextSymbol{\textunderscore}{TU}{`\_}
\DeclareTextSymbol{\textvisiblespace}{TU}{"2423}
\DeclareTextSymbol{\th}{TU}{"00FE}
\DeclareTextComposite{\.}{TU}{i}{`\i}
\DeclareTextComposite{\.}{TU}{\i}{`\i}
\DeclareTextComposite{\u}{TU}{A}{"0102}
\DeclareTextComposite{\k}{TU}{A}{"0104}
\DeclareTextComposite{\'}{TU}{C}{"0106}
\DeclareTextComposite{\v}{TU}{C}{"010C}
\DeclareTextComposite{\v}{TU}{D}{"010E}
\DeclareTextComposite{\v}{TU}{E}{"011A}
\DeclareTextComposite{\k}{TU}{E}{"0118}
\DeclareTextComposite{\u}{TU}{G}{"011E}
\DeclareTextComposite{\'}{TU}{L}{"0139}
\DeclareTextComposite{\v}{TU}{L}{"013D}
\DeclareTextComposite{\'}{TU}{N}{"0143}
\DeclareTextComposite{\v}{TU}{N}{"0147}
\DeclareTextComposite{\H}{TU}{O}{"0150}
\DeclareTextComposite{\'}{TU}{R}{"0154}
\DeclareTextComposite{\v}{TU}{R}{"0158}
\DeclareTextComposite{\'}{TU}{S}{"015A}
\DeclareTextComposite{\v}{TU}{S}{"0160}
\DeclareTextComposite{\c}{TU}{S}{"015F}
\DeclareTextComposite{\v}{TU}{T}{"0164}
\DeclareTextComposite{\c}{TU}{T}{"0162}
\DeclareTextComposite{\H}{TU}{U}{"0170}
\DeclareTextComposite{\r}{TU}{U}{"016E}
\DeclareTextComposite{\"}{TU}{Y}{"0178}
\DeclareTextComposite{\'}{TU}{Z}{"017A}
\DeclareTextComposite{\v}{TU}{Z}{"017D}
\DeclareTextComposite{\.}{TU}{Z}{"017B}
\DeclareTextComposite{\.}{TU}{I}{"0130}
\DeclareTextComposite{\u}{TU}{a}{"0103}
\DeclareTextComposite{\k}{TU}{a}{"0105}
\DeclareTextComposite{\'}{TU}{c}{"0107}
\DeclareTextComposite{\v}{TU}{c}{"010D}
\DeclareTextComposite{\v}{TU}{d}{"010F}
\DeclareTextComposite{\v}{TU}{e}{"011B}
\DeclareTextComposite{\k}{TU}{e}{"0119}
\DeclareTextComposite{\u}{TU}{g}{"011F}
\DeclareTextComposite{\'}{TU}{l}{"0139}
\DeclareTextComposite{\v}{TU}{l}{"013E}
\DeclareTextComposite{\'}{TU}{n}{"0144}
\DeclareTextComposite{\v}{TU}{n}{"0148}
\DeclareTextComposite{\H}{TU}{o}{"0151}
\DeclareTextComposite{\'}{TU}{r}{"0155}
\DeclareTextComposite{\v}{TU}{r}{"0159}
\DeclareTextComposite{\'}{TU}{s}{"015B}
\DeclareTextComposite{\v}{TU}{s}{"0161}
\DeclareTextComposite{\c}{TU}{s}{"015F}
\DeclareTextComposite{\v}{TU}{t}{"0165}
\DeclareTextComposite{\c}{TU}{t}{"0163}
\DeclareTextComposite{\H}{TU}{u}{"0171}
\DeclareTextComposite{\r}{TU}{u}{"016F}
\DeclareTextComposite{\"}{TU}{y}{"00FF}
\DeclareTextComposite{\'}{TU}{z}{"00FD}
\DeclareTextComposite{\v}{TU}{z}{"017E}
\DeclareTextComposite{\.}{TU}{z}{"017C}
\DeclareTextComposite{\`}{TU}{A}{"00C0}
\DeclareTextComposite{\'}{TU}{A}{"00C1}
\DeclareTextComposite{\^}{TU}{A}{"00C2}
\DeclareTextComposite{\~}{TU}{A}{"00C3}
\DeclareTextComposite{\"}{TU}{A}{"00C4}
\DeclareTextComposite{\r}{TU}{A}{"00C5}
\DeclareTextComposite{\c}{TU}{C}{"00C7}
\DeclareTextComposite{\`}{TU}{E}{"00C8}
\DeclareTextComposite{\'}{TU}{E}{"00C9}
\DeclareTextComposite{\^}{TU}{E}{"00CA}
\DeclareTextComposite{\"}{TU}{E}{"00CB}
\DeclareTextComposite{\`}{TU}{I}{"00CC}
\DeclareTextComposite{\'}{TU}{I}{"00CD}
\DeclareTextComposite{\^}{TU}{I}{"00CE}
\DeclareTextComposite{\"}{TU}{I}{"00CF}
\DeclareTextComposite{\~}{TU}{N}{"00D1}
\DeclareTextComposite{\`}{TU}{O}{"00D2}
\DeclareTextComposite{\'}{TU}{O}{"00D3}
\DeclareTextComposite{\^}{TU}{O}{"00D4}
\DeclareTextComposite{\~}{TU}{O}{"00D5}
\DeclareTextComposite{\"}{TU}{O}{"00D6}
\DeclareTextComposite{\`}{TU}{U}{"00D9}
\DeclareTextComposite{\'}{TU}{U}{"00DA}
\DeclareTextComposite{\^}{TU}{U}{"00DB}
\DeclareTextComposite{\"}{TU}{U}{"00DC}
\DeclareTextComposite{\'}{TU}{Y}{"00DD}
\DeclareTextComposite{\`}{TU}{a}{"00E0}
\DeclareTextComposite{\'}{TU}{a}{"00E1}
\DeclareTextComposite{\^}{TU}{a}{"00E2}
\DeclareTextComposite{\~}{TU}{a}{"00E3}
\DeclareTextComposite{\"}{TU}{a}{"00E4}
\DeclareTextComposite{\r}{TU}{a}{"00E5}
\DeclareTextComposite{\c}{TU}{c}{"00E7}
\DeclareTextComposite{\`}{TU}{e}{"00E8}
\DeclareTextComposite{\'}{TU}{e}{"00E9}
\DeclareTextComposite{\^}{TU}{e}{"00EA}
\DeclareTextComposite{\"}{TU}{e}{"00EB}
\DeclareTextComposite{\`}{TU}{i} {"00EC}
\DeclareTextComposite{\`}{TU}{\i}{"00EC}
\DeclareTextComposite{\'}{TU}{i} {"00ED}
\DeclareTextComposite{\'}{TU}{\i}{"00ED}
\DeclareTextComposite{\^}{TU}{i} {"00EE}
\DeclareTextComposite{\^}{TU}{\i}{"00EE}
\DeclareTextComposite{\"}{TU}{i} {"00EF}
\DeclareTextComposite{\"}{TU}{\i}{"00EF}
\DeclareTextComposite{\~}{TU}{n}{"00F1}
\DeclareTextComposite{\`}{TU}{o}{"00F2}
\DeclareTextComposite{\'}{TU}{o}{"00F3}
\DeclareTextComposite{\^}{TU}{o}{"00F4}
\DeclareTextComposite{\~}{TU}{o}{"00F5}
\DeclareTextComposite{\"}{TU}{o}{"00F6}
\DeclareTextComposite{\`}{TU}{u}{"00F9}
\DeclareTextComposite{\'}{TU}{u}{"00FA}
\DeclareTextComposite{\^}{TU}{u}{"00FB}
\DeclareTextComposite{\"}{TU}{u}{"00FC}
\DeclareTextComposite{\'}{TU}{y}{"00FD}
\DeclareTextComposite{\k}{TU}{o}{"01EB}
\DeclareTextComposite{\k}{TU}{O}{"01EA}

\DeclareTextComposite{\c}{TU}{G}{"0122}
\DeclareTextComposite{\c}{TU}{g}{"0123} % note this cedilla is above not below :)
\DeclareTextComposite{\c}{TU}{K}{"0136}
\DeclareTextComposite{\c}{TU}{k}{"0137}
\DeclareTextComposite{\c}{TU}{L}{"013B}
\DeclareTextComposite{\c}{TU}{l}{"013C}
\DeclareTextComposite{\c}{TU}{N}{"0145}
\DeclareTextComposite{\c}{TU}{n}{"0146}
\DeclareTextComposite{\c}{TU}{R}{"0156}
\DeclareTextComposite{\c}{TU}{r}{"0157}


%%%%%%%%%%%%%%%%%%%%%%%%%%%%%%%%%%%
%% TS1 symbols

\DeclareUnicodeAccent{\capitalcedilla}{"0327}
\DeclareUnicodeAccent{\capitalogonek}{"0328}
\DeclareUnicodeAccent{\capitalgrave}{"0300}
\DeclareUnicodeAccent{\capitalacute}{"0301}
\DeclareUnicodeAccent{\capitalcircumflex}{"0302}
\DeclareUnicodeAccent{\capitaltilde}{"0303}
\DeclareUnicodeAccent{\capitaldieresis}{"0308}
\DeclareUnicodeAccent{\capitalhungarumlaut}{"030B}
\DeclareUnicodeAccent{\capitalring}{"030A}
\DeclareUnicodeAccent{\capitalcaron}{"030C}
\DeclareUnicodeAccent{\capitalbreve}{"0306}
\DeclareUnicodeAccent{\capitalmacron}{"0304}
\DeclareUnicodeAccent{\capitaldotaccent}{"0307}
\DeclareUnicodeAccent{\t}{"0361}
\DeclareUnicodeAccent{\capitaltie}{"0361}
\DeclareUnicodeAccent{\newtie}{"0311}
\DeclareUnicodeAccent{\capitalnewtie}{"0311}
%%\DeclareTextSymbol{\textcapitalcompwordmark}{TU}{23}
%%\DeclareTextSymbol{\textascendercompwordmark}{TU}{31}
\DeclareTextSymbol{\textquotestraightbase}{TU}{"201A}
\DeclareTextSymbol{\textquotestraightdblbase}{TU}{"201E}
\DeclareTextSymbol{\texttwelveudash}{TU}{"2015}
\DeclareTextSymbol{\textthreequartersemdash}{TU}{"2012}
\DeclareTextSymbol{\textleftarrow}{TU}{"2190}
\DeclareTextSymbol{\textrightarrow}{TU}{"2192}
\DeclareTextSymbol{\textblank}{TU}{"2422}
\DeclareTextSymbol{\textdollar}{TU}{`\$}
\DeclareTextSymbol{\textquotesingle}{TU}{`\'}
%% \DeclareTextSymbol{\textasteriskcentered}{TU}{"002A}
%% \DeclareTextSymbol{\textdblhyphen}{TU}{45}
\DeclareTextSymbol{\textfractionsolidus}{TU}{"2044}
%%\DeclareTextSymbol{\textzerooldstyle}{TU}{48}
%%\DeclareTextSymbol{\textoneoldstyle}{TU}{49}
%%\DeclareTextSymbol{\texttwooldstyle}{TU}{50}
%%\DeclareTextSymbol{\textthreeoldstyle}{TU}{51}
%%\DeclareTextSymbol{\textfouroldstyle}{TU}{52}
%%\DeclareTextSymbol{\textfiveoldstyle}{TU}{53}
%%\DeclareTextSymbol{\textsixoldstyle}{TU}{54}
%%\DeclareTextSymbol{\textsevenoldstyle}{TU}{55}
%%\DeclareTextSymbol{\texteightoldstyle}{TU}{56}
%%\DeclareTextSymbol{\textnineoldstyle}{TU}{57}
\DeclareTextSymbol{\textlangle}{TU}{"3008}
\DeclareTextSymbol{\textminus}{TU}{"2212}
\DeclareTextSymbol{\textrangle}{TU}{"3009}
\DeclareTextSymbol{\textmho}{TU}{"2127}
\DeclareTextSymbol{\textbigcircle}{TU}{"25EF}
\DeclareUnicodeAccent{\textcircled}{"20DD}
\DeclareTextSymbol{\textohm}{TU}{"2126}
\DeclareTextSymbol{\textlbrackdbl}{TU}{"301A}
\DeclareTextSymbol{\textrbrackdbl}{TU}{"301B}
\DeclareTextSymbol{\textuparrow}{TU}{"2191}
\DeclareTextSymbol{\textdownarrow}{TU}{"2193}
\DeclareTextSymbol{\textasciigrave}{TU}{`\`}
\DeclareTextSymbol{\textborn}{TU}{"2605} %% actually "black star" by close enough
\DeclareTextSymbol{\textdivorced}{TU}{"26AE}
\DeclareTextSymbol{\textdied}{TU}{"2020} %% different from "dagger"??
%% \DeclareTextSymbol{\textleaf}{TU}{108}
\DeclareTextSymbol{\textmarried}{TU}{"26AD}
\DeclareTextSymbol{\textmusicalnote}{TU}{"266A}
\DeclareTextSymbol{\texttildelow}{TU}{"02F7}
%% \DeclareTextSymbol{\textdblhyphenchar}{TU}{127}
\DeclareTextSymbol{\textasciibreve}{TU}{"02D8}
\DeclareTextSymbol{\textasciicaron}{TU}{"02C7}
\DeclareTextSymbol{\textacutedbl}{TU}{"02DD}
\DeclareTextSymbol{\textgravedbl}{TU}{"02F5}
\DeclareTextSymbol{\textdagger}{TU}{"2020}
\DeclareTextSymbol{\textdaggerdbl}{TU}{"2021}
\DeclareTextSymbol{\textbardbl}{TU}{"2016}
\DeclareTextSymbol{\textperthousand}{TU}{"2030}
\DeclareTextSymbol{\textbullet}{TU}{"2022}
\DeclareTextSymbol{\textcelsius}{TU}{"2103}
%% \DeclareTextSymbol{\textdollaroldstyle}{TU}{138}
%% \DeclareTextSymbol{\textcentoldstyle}{TU}{139}
\DeclareTextSymbol{\textflorin}{TU}{"0192}
\DeclareTextSymbol{\textcolonmonetary}{TU}{"20A1}
\DeclareTextSymbol{\textwon}{TU}{"20A9}
\DeclareTextSymbol{\textnaira}{TU}{"20A6}
%% \DeclareTextSymbol{\textguarani}{TU}{144}
\DeclareTextSymbol{\textpeso}{TU}{"20B1}
\DeclareTextSymbol{\textlira}{TU}{"20A4}
\DeclareTextSymbol{\textrecipe}{TU}{"211E}
\DeclareTextSymbol{\textinterrobang}{TU}{"203D}
\DeclareTextSymbol{\textinterrobangdown}{TU}{"2E18}
\DeclareTextSymbol{\textdong}{TU}{"20AB}
\DeclareTextSymbol{\texttrademark}{TU}{"2122}
\DeclareTextSymbol{\textpertenthousand}{TU}{"2031}
\DeclareTextSymbol{\textpilcrow}{TU}{"00B6}
\DeclareTextSymbol{\textbaht}{TU}{"0E3F}
\DeclareTextSymbol{\textnumero}{TU}{"2116}
\DeclareTextSymbol{\textdiscount}{TU}{"2052}
\DeclareTextSymbol{\textestimated}{TU}{"212E}
\DeclareTextSymbol{\textopenbullet}{TU}{"25E6}
\DeclareTextSymbol{\textservicemark}{TU}{"2120}
\DeclareTextSymbol{\textlquill}{TU}{"2045}
\DeclareTextSymbol{\textrquill}{TU}{"2046}
\DeclareTextSymbol{\textcent}{TU}{"00A2}
\DeclareTextSymbol{\textsterling}{TU}{"00A3}
\DeclareTextSymbol{\textcurrency}{TU}{"00A3}
\DeclareTextSymbol{\textyen}{TU}{"00A5}
\DeclareTextSymbol{\textbrokenbar}{TU}{"00A6}
\DeclareTextSymbol{\textsection}{TU}{"00A7}
\DeclareTextSymbol{\textasciidieresis}{TU}{"00A8}
\DeclareTextSymbol{\textcopyright}{TU}{"00A9}
\DeclareTextSymbol{\textordfeminine}{TU}{"00AA}
%% \DeclareTextSymbol{\textcopyleft}{TU}{171}
\DeclareTextSymbol{\textlnot}{TU}{"00AC}
\DeclareTextSymbol{\textcircledP}{TU}{"2117}
\DeclareTextSymbol{\textregistered}{TU}{"00AE}
\DeclareTextSymbol{\textasciimacron}{TU}{"00AF}
\DeclareTextSymbol{\textdegree}{TU}{"00B0}
\DeclareTextSymbol{\textpm}{TU}{"00B1}
\DeclareTextSymbol{\texttwosuperior}{TU}{"00B2}
\DeclareTextSymbol{\textthreesuperior}{TU}{"00B3}
\DeclareTextSymbol{\textasciiacute}{TU}{"00B4}
\DeclareTextSymbol{\textmu}{TU}{"00B5}
\DeclareTextSymbol{\textparagraph}{TU}{"00B6}
\DeclareTextSymbol{\textperiodcentered}{TU}{"00B7}
\DeclareTextSymbol{\textreferencemark}{TU}{"203B}
\DeclareTextSymbol{\textonesuperior}{TU}{"00B9}
\DeclareTextSymbol{\textordmasculine}{TU}{"00BA}
\DeclareTextSymbol{\textsurd}{TU}{"221A}
\DeclareTextSymbol{\textonequarter}{TU}{"00BC}
\DeclareTextSymbol{\textonehalf}{TU}{"00BD}
\DeclareTextSymbol{\textthreequarters}{TU}{"00BE}
\DeclareTextSymbol{\texteuro}{TU}{"20AC}
\DeclareTextSymbol{\texttimes}{TU}{"00D7}
\DeclareTextSymbol{\textdiv}{TU}{"00F7}

\endinput

  \tl_set_eq:NN \UnicodeEncodingName \l_@@_unicode_name_tl
}

\DeclareDocumentCommand \ImportEncodingFile {m} {
  \bool_if:NF \l_@@_defining_encoding_bool { \ERROR }
  \clist_map_inline:nn {#1}
    {
      \InputIfFileExists{##1}{}{
        \@latex@error{Unicode~ encoding~ file~ `##1'~ not~ found}{\@ehd}%
      }
    }
}

% \DeclareUnicodeEncoding{user-legacy}{T1,TS1}{}

%    \end{macrocode}


% From 2e documenation:
% \begin{quote}
%    If you say:
%\begin{verbatim}
%    \DeclareTextCommand{\foo}{T1}...
%\end{verbatim}
%    then |\foo| is defined to be |\T1-cmd \foo \T1\foo|,
%    where |\T1\foo| is \emph{one} control sequence, not two!
% \end{quote}
%    \begin{macrocode}
\DeclareDocumentCommand \UndeclareSymbol {m}
  {
    \cs_undefine:c { \LastDeclaredEncoding \token_to_str:N #1 }
  }

\DeclareDocumentCommand \UndeclareComposite {mm}
  {
    \cs_undefine:c
      { \c_backslash_str \LastDeclaredEncoding \token_to_str:N #1 - \tl_to_str:n {#2} }
  }
%    \end{macrocode}
%

% \iffalse
%    \begin{macrocode}
%</fontspec&(xetexx|luatex)>
%    \end{macrocode}
% \fi
