% This package should work with any encoding name.
% Simply define an expansion for \UTFencname before loading this file,
% otherwise the encoding name will be 'U'.
\providecommand{\UTFencname}{U}
%
%
%
%
% Use \DeclareUTFcharacter to assign a cs-name to
% access a Unicode code-point...
%
\newcommand{\DeclareUTFcharacter}[3][\UTFencname]{%
  \let\add@flag\@ne % ==> add support in this encoding
  \check@hexcom@digits #2@@@@@!@{#1}{#2}{#3}%
}
%
% ... or use \UndeclareUTFcharacter to cancel a declaration
% when the appropriate code-point is not supported in the
% desired text-font.
%
\newcommand{\UndeclareUTFcharacter}[3][\UTFencname]{%
  \let\add@flag\z@ % ==> remove support in this encoding
  \check@hexcom@digits #2@@@@@!@{#1}{#2}{#3}%
}
%
%
%
\def\check@hexcom@digits#1#2@!@#3#4#5{%
 \ifx x#1\relax
  \check@hexcom@digits@#2@!@{#3}{#4}{#5}%
 \else
  \UTFacc@warning@{code #4 for #3-\string#5 fails to start with 'x'}%
 \fi
}
%
% Use \DeclareUTFcomposite to assign a cs-name to access
% accents or composite characters via Unicode code-points,
% or the Unicode "Composing Character" mechanism ...
%
\newcommand{\DeclareUTFcomposite}[4][\UTFencname]{{%
  \let\add@flag\@ne % ==> add support in this encoding
  \check@hex@digits #2@@@@@!@{#1}{#2}{#3}{#4}%
}}
\newcommand{\DeclareUTFmulticomposite}[4][\UTFencname]{{%
  \let\add@flag\@ne % ==> add support in this encoding
  \check@hex@digits #2@@@@@!@{#1}{#2}{#3}{#4}%
}}
%
% ... or use \UndeclareUTFcomposite to cancel a declaration
% when the appropriate code-point is not supported in the
% desired text-font.
%
\newcommand{\UndeclareUTFcomposite}[4][\UTFencname]{{%
  \let\add@flag\z@ % ==> remove support in this encoding
  \check@hex@digits #2@@@@@!@{#1}{#2}{#3}{#4}%
}}
%
% Allow already defined math-commands to be used also in text.
% Use of this is controlled by the 'mathastext' package option.
\newcommand\DeclareMathAsUTFtext[3]{%
 \expandafter\let\csname keepmathUTF#1\endcsname#2\relax
 \DeclareUTFcharacter[\UTFencname]{#3}{#2}%
 \expandafter\let\csname keeptextUTF#1\expandafter\endcsname
  \csname\UTFencname\string#2\endcsname\relax
 \DeclareTextCommand{#2}{\UTFencname}{%
  \ifmmode % the saved math version
   \expandafter\mathrm\csname keepmathUTF#1\endcsname
  \else   % the text version
   {\csname keeptextUTF#1\endcsname}%
 \fi}%
}
% useage: \DeclareMathAsUTFtext{aleph}{\aleph}{x2135}
%
%
%
%
\def\check@hex@digits#1#2@!@#3#4#5#6{%
 \ifx x#1\relax
   \check@hex@digits@#2@!@{#3}{#4}{#5}{#6}%
 \else
  \UTFacc@warning@{code #4 for #3-\string#5#6 fails to start with 'x'}%
 \fi
}
%
% indirect conditionals, to avoid unbalance when reloaded
\def\UTF@ignore#1{\csname iffalse\endcsname}
\def\UTF@doit#1{\csname iftrue\endcsname}
%
%%
%% these next macros need to have " with correct \catcode
%%
{\catcode`\"=12
%
\gdef\check@hexcom@digits@#1#2#3#4#5@!@#6#7#8{%
 \ifx @#4\relax
  \UTFacc@warning@{insufficient hex digits #7 for #6-\string#8}%
 \else
  \ifcat \active\noexpand#8%
   \ifx\add@flag\@ne %
    \expandafter\def\csname\UTFencname\string#8\endcsname{\char"#1#2#3#4\relax}%
    \ifx\unDeFiNed@#8%
     \ifx\cf@encoding\UTFencname
      \DeclareTextCommand{#8}{OT1}{\undefined}%
     \else
      \DeclareTextCommand{#8}{\cf@encoding}{\undefined}%
     \fi
    \else {% macro #8 exists already ...
      \let\protect\noexpand
      \edef\UTF@testi{#8}\def\UTF@testii{#8}%
      \ifx\UTF@testi\UTF@testii\aftergroup\UTF@ignore
      \else\aftergroup\UTF@doit\fi
     }%
     \iffalse
      % ... but when it isn't robust, make it so
      \expandafter\let\csname?-\string#8\endcsname#8\relax
      \edef\next@UTF@{{\cf@encoding}%
        {\expandafter\noexpand\csname?-\string#8\endcsname}}%
      \expandafter\DeclareTextCommand\expandafter
         {\expandafter#8\expandafter}\next@UTF@
     \fi
    \fi %
   \else % \add@flag \z@
    \expandafter\global\expandafter
      \let\csname\UTFencname\string#8\endcsname\relax
   \fi % end of \add@flag switch
  \else % not active catcode --- shouldn't happen
  % \typeout{*** did you really mean #8 ? ***}%
   \ifx\add@flag\@ne %
    \edef\tmp@name{\expandafter\string\csname\UTFencname\endcsname
      \expandafter\string\csname#8\endcsname}%
    \expandafter\def\csname\tmp@name\endcsname{\char"#1#2#3#4\relax}%
    \ifx\cf@encoding\UTFencname
     \expandafter\DeclareTextCommand\expandafter
       {\csname#8\endcsname}{OT1}{\undefined}%
    \else
     \expandafter\DeclareTextCommand\expandafter
       {\csname#8\endcsname}{\cf@encoding}{\undefined}%
    \fi
   \else % \add@flag \z@
    \expandafter\global\expandafter\let\csname#8\endcsname\relax
   \fi % end of \add@flag switch
  \fi % end of \ifcat
 \fi}
\gdef\check@hex@digits@#1#2#3#4#5@!@#6#7#8#9{%
 \ifx @#4\relax
  \UTFacc@warning@{insufficient hex digits #7 for #6-\string#8#9}%
 \else
  \def\UTFchar{\char"#1#2#3#4\relax}%
  \expandafter\expandafter\expandafter\declare@utf@composite
  \expandafter\expandafter\expandafter
   {\expandafter\csname#6\endcsname}{\UTFchar}{#8}{#9}\relax
 \fi}
%\gdef\add@UTF@accent#1#2#3{#2\char"#1\relax}
\gdef\add@UTF@accent#1#2#3{\ifx\relax#2\relax\char"#3\else
 \ifx\ #2\relax\char"#3\else
 \expandafter\ifx\UTF@space#2\relax\char"#3\else
 \ifx~#2\char"#3\else#2\char"#1\fi\fi\fi\fi\relax}
\gdef\add@UTF@accents#1#2#3{#2\char"#1\char"#3\relax}
\gdef\add@set@accentCOMP#1#2#3{\add@accent{"#1}{#2}}
\gdef\add@set@accentMOD#1#2#3{\add@accent{"#3}{#2}}
\gdef\declare@hex@command#1#2{\gdef#2{#1}}%
%
}%  end of \catcode`\"=12
%
{\catcode`\ =10\relax%
\gdef\UTF@@space{ }}%
\edef\UTF@space{\UTF@@space}
%
\def\declare@utf@composite#1#2#3#4{%
 \expandafter\ifcat\expandafter A\string#4\relax
  {\ifx\add@flag\@ne %
   \expandafter\xdef\csname\string#1\string#3-#4\endcsname{#2}%
  \else
   \expandafter\global\expandafter
    \let\csname\string#1\string#3-#4\endcsname\relax
  \fi}%
 \else
  {\ifx\add@flag\@ne %
   \expandafter\xdef\csname\string#1\string#3-\string#4\endcsname{#2}%
  \else
   \expandafter\global\expandafter
    \let\csname\string#1\string#3-\string#4\endcsname\relax
  \fi}%
 \fi
}
%
% new command:  {\DeclareEncodedCompositeCharacter}[4]{%
  %  #1 = encoding
  %  #2 = accent-macro in TeX
  %  #3 = position of combining glyph in Unicode
  %  #4 = bare accent position, in Unicode
  %  ##1 = slot for the accented letter
%\newcommand{\DeclareEncodedCompositeCharacter}[4]{%
%  \expandafter\def\csname #1\string#2\endcsname##1{%
%    \expandafter\@text@composite \csname #1\string#2\endcsname##1\@empty
%    \@text@composite{\add@encoded@accent{#3}{##1}{#4}}}%
%}
\newcommand{\DeclareEncodedCompositeCharacter}[4]{%
  \expandafter\def\expandafter\next@ii\expandafter{%
   \expandafter\expandafter\expandafter\@text@composite\expandafter
    \csname #1\string#2\endcsname####1\@empty
     \@text@composite{\add@encoded@accent{#3}{####1}{#4}}}%
  \expandafter\def\expandafter\next@i\expandafter{\expandafter\expandafter
   \expandafter\def\expandafter\csname #1\string#2\endcsname####1}%
  \expandafter\next@i\expandafter{\next@ii}%
}
%\newcommand{\DeclareEncodedCompositeAccents}[4]{%
%  \expandafter\def\csname #1\string#2\endcsname##1{%
%    \expandafter\@text@composite \csname #1\string#2\endcsname##1\@empty
%    \@text@composite{\add@encoded@accents{#4}{##1}{#3}}}%
%}
\newcommand{\DeclareEncodedCompositeAccents}[4]{%
  \expandafter\def\expandafter\next@ii\expandafter{%
   \expandafter\expandafter\expandafter\@text@composite\expandafter
    \csname #1\string#2\endcsname####1\@empty
     \@text@composite{\add@encoded@accent{#4}{####1}{#3}}}%
  \expandafter\def\expandafter\next@i\expandafter{\expandafter\expandafter
   \expandafter\def\expandafter\csname #1\string#2\endcsname####1}%
  \expandafter\next@i\expandafter{\next@ii}%
}

\let\add@encoded@accent\add@UTF@accent
\let\add@encoded@accents\add@UTF@accents
%\let\add@encoded@accent\add@set@accentCOMP
%\let\add@encoded@accent\add@set@accentMOD
%
% bring \textsuperscript and \textsubscript into the fold of macros 
% dependent on encoding
%
\let\realLaTeXsuperscript\textsuperscript
\let\realLaTeXsubscript\textsubscript
\DeclareTextAccent{\textsuperscript}{OT1}{999}
\expandafter\expandafter\expandafter\let\expandafter
 \csname?\string\textsuperscript\endcsname\realLaTeXsuperscript
\DeclareTextAccent{\textsubscript}{OT1}{999}
\expandafter\expandafter\expandafter\let\expandafter
 \csname?\string\textsubscript\endcsname\realLaTeXsubscript
\let\super\textsuperscript
%

%: separator

% need to patch the accents for use with TIPA's T3 encoded letters
% and shorthand for double-accents

\input t3enc.def

\let\realLaTeXacute\'    \def\tipaacuteaccent{\TIPAaccent{\realLaTeXacute}}
\let\realLaTeXgrave\`    \def\tipagraveaccent{\TIPAaccent{\realLaTeXgrave}}
\let\realLaTeXcircum\^   \def\tipacircumaccent{\TIPAaccent{\realLaTeXcircum}}
\let\realLaTeXumlaut\"   \def\tipaumlautaccent{\TIPAaccent{\realLaTeXumlaut}}
\let\realLaTeXmacron\=   \def\tipamacronaccent{\TIPAaccent{\realLaTeXmacron}}
\let\realLaTeXtilde\~    \def\tipatildeaccent{\TIPAaccent{\realLaTeXtilde}}
\let\realLaTeXdot\.      \def\tipadotaccent{\TIPAaccent{\realLaTeXdot}}
\let\realLaTeXring\r     \def\tiparingaccent{\TIPAaccent{\realLaTeXring}}
\let\realLaTeXbreve\u    \def\tipabreveaccent{\TIPAaccent{\realLaTeXbreve}}
\let\realLaTeXcaron\v    \def\tipacaronaccent{\TIPAaccent{\realLaTeXcaron}}
\let\realLaTeXogonek\k   \def\tipaogonekaccent{\TIPAaccent{\realLaTeXogonek}}
\let\realLaTeXhungar\H   \def\tipahungaraccent{\TIPAaccent{\realLaTeXhungar}}
\let\realLaTeXcedilla\c  \def\tipacedillaaccent{\TIPAaccent{\realLaTeXcedilla}}
\let\realLaTeXmisc\m     \def\tipamiscaccent{\TIPAaccent{\realLaTeXmisc}}
\let\realLaTeXtie\t      \def\tipatieaccent{\TIPAaccent{\realLaTeXtie}}
\let\realLaTeXvert\|     \def\tipavertaccent{\TIPAaccent{\TIPAvertmacro}}%

\let\realLaTeXmathcolon\:      \def\tipacolonmacro{%
  \ifmmode\expandafter\realLaTeXmathcolon\else\expandafter\TIPAcolonmacro\fi}
\let\realLaTeXmathsemicolon\:  \def\tipasemicolonmacro{%
  \ifmmode\expandafter\realLaTeXmathsemicolon\else\expandafter\TIPAsemicolonmacro\fi}
\let\realLaTeXmathstar\*       \def\tipastarmacro{%
  \ifmmode\expandafter\realLaTeXmathstar\else\expandafter\TIPAstarmacro\fi}
\let\realLaTeXmathexclam\!     \def\tipaexclammacro{%
  \ifmmode\expandafter\realLaTeXmathexclam\else\expandafter\TIPAexclammacro\fi}
%\let\realLaTeXmathvert\|        \def\tipavertmacro{%
%  \ifmmode\expandafter\realLaTeXmathvert\else\expandafter\TIPAvertmacro\fi}


\DeclareTextAccent{\TIPAcolonmacro}{OT1}{999}
\DeclareTextAccent{\TIPAsemicolonmacro}{OT1}{999}
\DeclareTextAccent{\TIPAstarmacro}{OT1}{999}
\DeclareTextAccent{\TIPAexclammacro}{OT1}{999}
\DeclareTextAccent{\TIPAvertmacro}{OT1}{999}

\def\setupTIPAaccents{%
  \let\'\tipaacuteaccent
  \let\^\tipacircumaccent
  \let\`\tipagraveaccent
  \let\~\tipatildeaccent
  \let\=\tipamacronaccent
  \let\"\tipaumlautaccent
  \let\.\tipadotaccent
  \let\r\tiparingaccent
  \let\k\tipaogonekaccent
  \let\c\tipacedillaaccent
  \let\u\tipabreveaccent
  \let\v\tipawedgeaccent
  \let\H\tipahungaraccent
  \let\t\tipatieaccent
  \let\m\tipamiscaccent
  \let\|\tipavertaccent
%
  \let\s\textsyllabic
  \let\:\tipacolonmacro
  \let\;\tipasemicolonmacro
  \let\*\tipastarmacro
  \let\!\tipaexclammacro
}


\def\TIPAaccent#1#2{\bgroup
  \def\donextaccent{\egroup#1#2}%
%  \def\doexpandaccent{\egroup\expandafter#1#2}%
  \def\doexpandaccentchar{\egroup\expandafter#1#2}%
  \def\doexpandaccentgroup{\egroup\expandafter#1\expandafter{#2}}%
  \def\tmpa{#2}\expandafter\def\expandafter\tmpb\expandafter{#2}%
  \ifx\tmpa\tmpb\else
    % have to do more here for nested accents !!! 
    \expandafter\def\expandafter\tmp\expandafter{\string#2}%
    \expandafter\testforslash\tmp\$!\$%
  \fi \donextaccent }
\def\testforslash#1#2\$!\${\def\tmp{#1}%
  \ifx\tmp\TIPAbareslash
   \ifx\relax#2\relax\else
    % for nested accents or macros
    \let\donextaccent\doexpandaccentgroup
   \fi
  \else
    % for active characters
   \let\donextaccent\doexpandaccentchar
  \fi}
\def\catchbareslash#1#2\$!\${\def\TIPAbareslash{#1}}
\edef\next{\string\ }
\expandafter\catchbareslash\next\$!\$


% This assumes that the shorthands for double-accents
% are used only within  \tipatext{...} portions:

\def\tipasubacuteaccent{\TIPAaccent{\textsubacute}}
\def\tipadotacuteaccent{\TIPAaccent{\textdotacute}}
\def\tipaacutemacronaccent{\TIPAaccent{\textacutemacron}}
\def\tipasubgraveaccent{\TIPAaccent{\textsubgrave}}
\def\tipasubcircumaccent{\TIPAaccent{\textsubcircum}}
\def\tipasubtildeaccent{\TIPAaccent{\textsubtilde}}
\def\tipasubdotaccent{\TIPAaccent{\textsubdot}}
\def\tipacircumdotaccent{\TIPAaccent{\textcircumdot}}
\def\tipatildedotaccent{\TIPAaccent{\texttildedot}}
\def\tipagravedotaccent{\TIPAaccent{\textgravedot}}
\def\tipagravemacronaccent{\TIPAaccent{\textgravemacron}}
\def\tipagravecircumaccent{\TIPAaccent{\textgravecircum}}
\def\tipasubbaraccent{\TIPAaccent{\textsubbar}}
\def\tipasubringaccent{\TIPAaccent{\textsubring}}
\def\tipasubwedgeaccent{\TIPAaccent{\textsubwedge}}
\def\tipasubumlautaccent{\TIPAaccent{\textsubumlaut}}
\def\tipadoublegraveaccent{\TIPAaccent{\textdoublegrave}}
\def\tiparingmacronaccent{\TIPAaccent{\textringmacron}}
\def\tipabrevemacronaccent{\TIPAaccent{\textbrevemacron}}
\def\tipaacutewedgeaccent{\TIPAaccent{\textacutewedge}}

\def\tipasubbridgeaccent{\TIPAaccent{\textsubbridge}}
\def\tipainvsubbridgeaccent{\TIPAaccent{\textinvsubbridge}}
\def\tipasublhalfringaccent{\TIPAaccent{\textsublhalfring}}
\def\tipasubrhalfringaccent{\TIPAaccent{\textsubrhalfring}}
\def\tiparoundcapaccent{\TIPAaccent{\textroundcap}}
\def\tipasubplusaccent{\TIPAaccent{\textsubplus}}
\def\tiparaisingaccent{\TIPAaccent{\textraising}}
\def\tipaloweringaccent{\TIPAaccent{\textlowering}}
\def\tipaadvancingaccent{\TIPAaccent{\textadvancing}}
\def\tiparetractingaccent{\TIPAaccent{\textretracting}}
\def\tipaovercrossaccent{\TIPAaccent{\textovercross}}
\def\tipasubwaccent{\TIPAaccent{\textsubw}}
\def\tipaundertieaccent{\TIPAaccent{\textundertie}}
\def\tipasubarchaccent{\TIPAaccent{\textsubarch}}
\def\tipaseagullaccent{\TIPAaccent{\textseagull}}

\newcount\tipaTiiicode
\renewcommand{\tipaloweraccent}[2][]{%
 \def\next{}%
 \tipaTiiicode=#2\relax
 \ifcase\tipaTiiicode\relax
  \notipaloweraccent{#2}%      % 0 L    0 U = grave
 \or\notipaloweraccent{#2}%    % 1 U = acute
 \or\let\next\textsubcircum % 2 L    2 U = circum
 \or\let\next\textsubtilde  % 3 L    3 U = tilde
 \or\let\next\textsubumlaut % 4 L    4 U = umlaut
 \or\notipaloweraccent{#2}%    % 5 U = hungar-umlaut
 \or\let\next\textsubring   % 6 L    6 U = ring
 \or\let\next\textsubwedge  % 7 L    7 U = caron
 \or\let\next\tipasubarch   % 8 L    8 U = breve
 \or\let\next\textsubbar    % 9 L    9 U = macron
 \or\let\next\textsubdot    % 10 L  10 U = dot
 \or\let\next\c             % 11 L  cedilla
 \or\let\next\textpolhook   % 12 L  ogonek
 \or\let\next\textdoublegrave % 13 U = textdoublegrave
 \or\let\next\textsubgrave  % 14 L
 \or\let\next\textsubacute  % 15 L
 \or\notipaloweraccent{#2}%    % 16 U = textroundcap
 \or\let\next\textsubarch   % 16 L
 \or\let\next\textsubbridge % 17 L
 \or\let\next\textinvsubbridge % 18 L
 \or\let\next\textsubsquare % 19 L
 \or\let\next\textsubrhalfring % 20 L
 \or\let\next\textsublhalfring % 21 L
 \or\let\next\textsubw      % 22 L
% \or\let\next\textoverw     % 22 U = textoverw
 \or\let\next\textseagull   % 23 L
 \or\notipaloweraccent{#2}%    % 24 U = textovercross
 \or\notipaloweraccent{#2}%    % 25
 \or\notipaloweraccent{#2}%    % 26
 \or\let\next\textsubplus   % 27 L
 \or\let\next\textraising   % 28 L
 \or\let\next\textlowering  % 29 L
 \or\let\next\textadvancing % 30 L
 \or\let\next\textretracting% 31 L
 \or\notipaloweraccent{#2}%    % 32
 \or\notipaloweraccent{#2}%    % 33
 \or\let\next\textsyllabic  % 34 L
 \or\expandafter % 35
 \or\expandafter % 36
 \or\expandafter % 37
 \or\let\next\textsuperimposetilde % 38 lap
 \or\expandafter % 39
 \or\expandafter % 4
 \or\expandafter % 5
 \or\let\next\textundertieaccent % 60 L
 \or\let\next\texttoptiebar    % 62 U
 \or\let\next\textmidacute     % 152 U
 \or\let\next\textgravemid     % 153 U
 \or\let\next\textgravecircum  % 154 U
 \or\let\next\textcircumacute  % 155 U
 \or\let\next\textvbaraccent   % 156 U
 \or\let\next\textdoublevbaraccent % 157 U
 \or\let\next\textgravedot     % 158 U
 \or\let\next\textdotacute     % 159 U
% 128 + 24 = 152
 \else
  \expandafter\def\expandafter\next\expandafter{%
   \expandafter\tipaxloweraccent\expandafter{\the\tipaTiiicode}}%
 \fi
 \next }
\renewcommand{\tipaupperaccent}[2][]{%
 \def\next{}%
 \tipaTiiicode=#2\relax
 \ifcase\tipaTiiicode\relax 
  \let\next\`%    0  grave accent
 \or\let\next\'%  1  acute accent
 \or\let\next\^%  2  circum accent
 \or\let\next\~%  3  tilde accent
 \or\let\next\"%  4  umlaut accent
 \or\let\next\H%  5  hungar-umlaut accent
 \or\let\next\r%  6  ring accent
 \or\let\next\v%  7  caron accent
 \or\let\next\u%  8  breve accent
 \or\let\next\=%  9  macron accent
 \or\let\next\.%  10 dot-above accent
 \or\notipaupperaccent{#2}%  11
 \or\notipaupperaccent{#2}%  12
 \or\notipaupperaccent{#2}%  13
 \or\notipaupperaccent{#2}%  14
 \or\notipaupperaccent{#2}%  15
 \or\let\next\textroundcap % 16
 \or\notipaupperaccent{#2}%  17
 \or\notipaupperaccent{#2}%  18
 \or\notipaupperaccent{#2}%  19
 \or\notipaupperaccent{#2}%  20
 \or\notipaupperaccent{#2}%  21
 \or\let\next\textoverw %    22
 \or\notipaupperaccent{#2}%  23
 \or\let\next\textovercross% 24 
 \else
  \expandafter\def\expandafter\next\expandafter{%
   \expandafter\tipaxupperaccent\expandafter{\the\tipaTiiicode}}%
 \fi
 \next }

\def\notipaloweraccent#1{\typeout{*** no lowerable accent in position #1 ***}}
\def\notipaupperaccent#1{\typeout{*** no raiseable accent in position #1 ***}}

\let\realtipaupperaccent\tipaUpperaccent
\def\supsdimi{.2ex}
\def\supsdimii{.8ex}
\providecommand{\sups}[2]{{%\tracingall
 \textipa{\realtipaupperaccent[\supsdimi]%
  {\lower\supsdimii\hbox{\super{#2}}}{#1}}}}

% Constructed multiple accents
\def\tipaDOTacuteaccent#1{{\.{\ }\kern-.25em#1\kern-.2em\'{\ }}}
\def\tipagraveDOTaccent#1{{\`{\ }\kern-.25em#1\kern-.2em\.{\ }}}
\AtBeginDocument{%
 \let\textdotacute\tipaDOTacuteaccent
 \let\textgravedot\tipagraveDOTaccent
}


%: separator


% must not declare this, as the encoding is picked up by \textsc 
\newcommand{\faketextsc}[1]{\scalebox{.7}[.7]{#1}}
\def\TIPAtextscq{{\faketextsc{Q}}}
%\providecommand{\textscf}{\faketextsc{F}}% UxA730
\providecommand{\textscq}{\faketextsc{Q}}%
\providecommand{\textscdelta}{\faketextsc{\textDelta}}%

% Here's a way to implement TIPA's T3 encoding for uppercase letters and digits
% It's not the best possible way to implement this; but it is as good as can be
% done in macros, without introducing token-by-token parsing.

\def\setTIPAcatcodes{%
  \catcode `A = \active
  \catcode `B = \active
  \catcode `C = \active
  \catcode `D = \active
  \catcode `E = \active
  \catcode `F = \active
  \catcode `G = \active
  \catcode `g = \active
  \catcode `H = \active
  \catcode `I = \active
  \catcode `J = \active
  \catcode `K = \active
  \catcode `L = \active
  \catcode `M = \active
  \catcode `N = \active
  \catcode `O = \active
  \catcode `P = \active
  \catcode `Q = \active
  \catcode `R = \active
  \catcode `S = \active
  \catcode `T = \active
  \catcode `U = \active
  \catcode `V = \active
  \catcode `W = \active
  \catcode `X = \active
  \catcode `Y = \active
  \catcode `Z = \active
  \catcode `0 = \active
  \catcode `1 = \active
  \catcode `2 = \active
  \catcode `3 = \active
  \catcode `4 = \active
  \catcode `5 = \active
  \catcode `6 = \active
  \catcode `7 = \active
  \catcode `8 = \active
  \catcode `9 = \active
  \catcode `\@ = \active
  \catcode `\; = \active
  \catcode `\: = \active
  \catcode `\" = \active
  \catcode `| = \active
}
% these catcodes need to be unset, when following \*
\def\unsetTIPAcatcodes{%
  \catcode `A = 11
  \catcode `B = 11
  \catcode `C = 11
  \catcode `D = 11
  \catcode `E = 11
  \catcode `F = 11
  \catcode `G = 11
  \catcode `g = 11
  \catcode `H = 11
  \catcode `I = 11
  \catcode `J = 11
  \catcode `K = 11
  \catcode `L = 11
  \catcode `M = 11
  \catcode `N = 11
  \catcode `O = 11
  \catcode `P = 11
  \catcode `Q = 11
  \catcode `R = 11
  \catcode `S = 11
  \catcode `T = 11
  \catcode `U = 11
  \catcode `V = 11
  \catcode `W = 11
  \catcode `X = 11
  \catcode `Y = 11
  \catcode `Z = 11
  \catcode `0 = 12
  \catcode `1 = 12
  \catcode `2 = 12
  \catcode `3 = 12
  \catcode `4 = 12
  \catcode `5 = 12
  \catcode `6 = 12
  \catcode `7 = 12
  \catcode `8 = 12
  \catcode `9 = 12
  \catcode `\@ = 12
  \catcode `\; = 12
  \catcode `\: = 12
  \catcode `\" = 12
  \catcode `| = 12
}

{\global\let\setuptipaaccents\setupTIPAaccents
 \def\next{\textscriptg}%
 \def\nextii{\catcode `g = 11 }%
 \catcode `g = \active 
 \catcode `\" = \active 
 \expandafter\def\expandafter\next\expandafter{%
  \expandafter\def\expandafter g\expandafter{\next}}
 \setTIPAcatcodes
 \nextii
 \let\zdef\gdef
 \expandafter\zdef\expandafter\activatetipa\expandafter{%
  \expandafter\setuptipaaccents\next 
 \def A{\textscripta}%
% \def B{\textbeta}%  name taken for the greek letter
 \def B{\ss}%
 \def C{\textctc}%
 \def D{\dh}%
 \def E{\textepsilon}%
 \def F{\textphi}%
% \def G{\textgamma}%
 \def G{\textbabygamma}%
 \def H{\texthth}%
 \def I{\textsci}%
 \def J{\textctj}%
 \def K{\textinvscr}%
 \def L{\textturny}%
 \def M{\textltailm}%
 \def N{\ng}%
 \def O{\textopeno}%
 \def P{\textglotstop}%
 \def Q{\textrevglotstop}%
 \def R{\textfishhookr}%
 \def S{\textesh}%
 \def T{\texttheta}%
% \def U{\textupsilon}%
 \def U{\textscupsilon}%
 \def V{\textscriptv}%
 \def W{\textturnm}%
 \def X{\textchi}%
 \def Y{\textscy}%
% \def Z{\textyogh}%
 \def Z{\textezh}%
 \def 0{\textbaru}%
 \def 1{\textbari}%
 \def 2{\textturnv}%
 \def 3{\textrevepsilon}%
 \def 4{\textturnh}%
 \def 5{\textturna}%
 \def 6{\textturnscripta}%
 \def 7{\textramshorns}%
 \def 8{\textbaro}%
 \def 9{\textreve}%
 \def @{\textschwa}%
 \def :{\textlengthmark}%
 \def ;{\texthalflength}%
 \def |{\textpipe}%
 \let "\tipaprimstress
 }
}%  end of  \setTIPAcatcodes
%\show\activatetipa

% Primary and Secondary stress shortcuts
\def\tipaprimstress{\futurelet\next\tipaprimstressi}
\def\tipaprimstressi{%
 \ifx\next\tipaprimstress
  \def\next##1{\textsecstress}%
 \else
  \def\next{\textprimstress}%
 \fi \next
}


% \textipa really needs to parse all tokens,
% otherwise many macros cannot be used effectively
\def\tipanoendline{\endlinechar=-1}
\def\tipalettercatcode{11}
\def\tipacatchg#1{{%
 \def\tmpa{#1}\def\tmpb{g}%
 \ifx\tmpa\tmpb \def\next{\textscriptg}%
 \else
  \def\next{\implementTIPAtext{%
   \catcode`g \tipalettercatcode\relax
    \tipanoendline\relax
    \scantokens{#1}%
  }}%
 \fi \next
}}

% To set a special font; e.g. Doulos SIL
% use standard techniques to define the font-switch
% then redefine  \useTIPAfont  as follows:
%   \def\useTIPAfont{\doulos}
\def\useTIPAfont{} 

\DeclareRobustCommand{\implementTIPAtext}{%
 \bgroup
  \let\rtone\TIPArtonebar
  \let\stone\TIPAstonebar
  \let\tone\TIPAtonebar
%  \setTIPAcatcodes
  \activatetipa
  \useTIPAfont
  \implementTIPAtextx
}
\def\implementTIPAtextx#1{#1\egroup}
\def\implementTIPAtextxx#1{\endlinechar=-1 \scantokens{#1}\egroup}
\AtBeginDocument{%
 \let\textipa\tipacatchonechar %\tipacatchg %\implementTIPAtext
 \let\rtone\TIPArtonebar
 \let\stone\TIPAstonebar
 \let\tone\TIPAtonebar
 }
\expandafter\ifx\csname scantokens\endcsname\relax
\else
 \AtBeginDocument{\@ifpackageloaded{linguex}{%
  \let\implementTIPAtextx\implementTIPAtextxx
  }{}}%
\fi

% this is used with \t and \sliding
\def\checkfortipa#1#2#3{%
 \ifx\textipa#3\def\next##1{\tipacatchonechar{#1{##1}}}%
 \else\def\next{#2#3}\fi \next }


\newtoks\tipasavetokens
\newtoks\tipachecktokens
\newif\iftipaonetoken
\def\tipalasttoken{!@! do nothing with this !@!}
\def\tipacatchonechar#1{\begingroup
 \def\textipa##1{##1}% prevent recursion
 %\let\implementTIPAtextx\implementTIPAtextxx % forces \scantokens
 \tipaonetokentrue
 \let\s\textsyllabic
 \tipasavetokens={#1}\tipatestforonechar#1\tipalasttoken}
\def\tipatestforonechar{\futurelet\next\tipatesttoken}
\def\tipatesttoken{%
 \def\next@##1{\tipaaddtoken{##1}\tipatestforonechar}%
 \ifx\next\tipalasttoken
  \def\next@##1{\tipaprocessthetokens}%
 \else\ifx\bgroup\next \tipaonetokenfalse
  \def\next@##1{\tipaparsegrouping{##1}\tipatestforonechar}%
 \else\expandafter\ifx\space\next
  \expandafter\def\expandafter\next@\space{\tipaaddtoken{ }\tipatestforonechar}%
 \else\ifx\*\next \def\next@{\gettipastar}%
 \else\ifx\;\next \def\next@{\gettipasemicolon}%
 \else\ifx\:\next \def\next@{\gettipacolon}%
 \else\ifx\!\next \def\next@{\gettipaexclam}%
 \else\ifx\|\next \def\next@{\gettipavert}%
 \else\ifx\tone\next \def\next@##1##2{\tipaaddtoken{##1{##2}}\tipatestforonechar}%
 \else\ifx\rtone\next \def\next@##1##2{\tipaaddtoken{##1{##2}}\tipatestforonechar}%
 \else\ifx\stone\next \def\next@##1##2{\tipaaddtoken{##1{##2}}\tipatestforonechar}%
 \else\expandafter\ifcat\noexpand\next\active\tipaonetokenfalse
 \else\ifcat\next A\relax % letter
  \def\next@{\gettipaletter}%
 \else\ifcat\next 0\relax % other character
  \def\next@{\gettipaother}%
 \else \tipaonetokenfalse
 \fi\fi\fi\fi\fi\fi\fi\fi\fi\fi\fi\fi\fi\fi
 \next@
}
\def\tipaaddtoken#1{\expandafter\tipachecktokens\expandafter{\the\tipachecktokens#1}}
\def\tipaprocessthetokens{%
% \iftipaonetoken
%  \expandafter\def\expandafter\next\expandafter{\expandafter
%   \implementTIPAtext\expandafter{\the\tipachecktokens}\endgroup}%
% \else
%  \expandafter\def\expandafter\next\expandafter{\expandafter
%   \tipacatchg\expandafter{\the\tipachecktokens}\endgroup}%
% \fi  \next
 \expandafter\implementTIPAtext\expandafter{\the\tipachecktokens}\endgroup
}
\def\tipaparsegrouping#1{\begingroup
 \tipachecktokens={}%
 \let\tipaprocessthetokens\relax
 \tipatestforonechar#1\tipalasttoken
 \expandafter\def\expandafter\next\expandafter{\expandafter\endgroup
  \expandafter\tipaaddtoken\expandafter{\expandafter{\the\tipachecktokens}}}%
% \show\next
 \next
}
\def\gettipaletter#1{%
 \ifx A#1\tipaaddtoken{\textscripta}%
 \else\ifx B#1\tipaaddtoken{\ss}%
 \else\ifx C#1\tipaaddtoken{\textctc}%
 \else\ifx D#1\tipaaddtoken{\dh}%
 \else\ifx E#1\tipaaddtoken{\textepsilon}%
 \else\ifx F#1\tipaaddtoken{\textphi}%
 \else\ifx G#1\tipaaddtoken{\textbabygamma}%
 \else\ifx g#1\tipaaddtoken{\textscriptg}%
 \else\ifx H#1\tipaaddtoken{\texthth}%
 \else\ifx I#1\tipaaddtoken{\textsci}%
 \else\ifx J#1\tipaaddtoken{\textctj}%
 \else\ifx K#1\tipaaddtoken{\textinvscr}%
 \else\ifx L#1\tipaaddtoken{\textturny}%
 \else\ifx M#1\tipaaddtoken{\textltailm}%
 \else\ifx N#1\tipaaddtoken{\ng}%
 \else\ifx O#1\tipaaddtoken{\textopeno}%
 \else\ifx P#1\tipaaddtoken{\textglotstop}%
 \else\ifx Q#1\tipaaddtoken{\textrevglotstop}%
 \else\ifx R#1\tipaaddtoken{\textfishhookr}%
 \else\ifx S#1\tipaaddtoken{\textesh}%
 \else\ifx T#1\tipaaddtoken{\texttheta}%
 \else\ifx U#1\tipaaddtoken{\textscupsilon}%
 \else\ifx V#1\tipaaddtoken{\textscriptv}%
 \else\ifx W#1\tipaaddtoken{\textturnm}%
 \else\ifx X#1\tipaaddtoken{\textchi}%
 \else\ifx Y#1\tipaaddtoken{\textscy}%
 \else\ifx Z#1\tipaaddtoken{\textyogh}%
 \else\tipaaddtoken{#1}%
 \fi\fi\fi\fi\fi\fi\fi\fi\fi\fi\fi\fi\fi\fi
 \fi\fi\fi\fi\fi\fi\fi\fi\fi\fi\fi\fi\fi
 \tipatestforonechar
}
{\catcode `\" 12 \catcode `\@ 12
\gdef\gettipaother#1{%
 \ifx @#1\tipaaddtoken{\textschwa}%
 \else\ifx 0#1\tipaaddtoken{\textbaru}%
 \else\ifx 1#1\tipaaddtoken{\textbari}%
 \else\ifx 2#1\tipaaddtoken{\textturnv}%
 \else\ifx 3#1\tipaaddtoken{\textrevepsilon}%
 \else\ifx 4#1\tipaaddtoken{\textturnh}%
 \else\ifx 5#1\tipaaddtoken{\textturna}%
 \else\ifx 6#1\tipaaddtoken{\textturnscripta}%
 \else\ifx 7#1\tipaaddtoken{\textramshorns}%
 \else\ifx 8#1\tipaaddtoken{\textbaro}%
 \else\ifx 9#1\tipaaddtoken{\textreve}%
 \else\ifx :#1\tipaaddtoken{\textlengthmark}%
 \else\ifx ;#1\tipaaddtoken{\texthalflength}%
 \else\ifx |#1\tipaaddtoken{\textpipe}%
 \else\ifx "#1\tipaaddtoken{\tipaprimstress}%
 \else\tipaaddtoken{#1}%
 \fi\fi\fi\fi\fi\fi\fi\fi\fi\fi\fi\fi\fi\fi\fi
 \tipatestforonechar
}}
\def\gettipastar\*#1{%
 \ifx f#1\tipaaddtoken{\textObardotlessj}%
 \else\ifx k#1\tipaaddtoken{\textturnk}%
 \else\ifx r#1\tipaaddtoken{\textturnr}%
 \else\ifx t#1\tipaaddtoken{\textturnt}%
 \else\ifx w#1\tipaaddtoken{\textturnw}%
 \else\ifx j#1\tipaaddtoken{\textbardotlessj}%
 \else\ifx n#1\tipaaddtoken{\textltailn}%
 \else\ifx h#1\tipaaddtoken{\texthbar}%
 \else\ifx l#1\tipaaddtoken{\textbeltl}%
 \else\ifx z#1\tipaaddtoken{\textlyoghlig}%
 \else\tipaaddtoken{#1}%
 \fi\fi\fi\fi\fi\fi\fi\fi\fi\fi
 \tipatestforonechar
}
\def\gettipasemicolon\;#1{%
 \ifx A#1\tipaaddtoken{\textsca}%
 \else\ifx B#1\tipaaddtoken{\textscb}%
 \else\ifx E#1\tipaaddtoken{\textsce}%
 \else\ifx G#1\tipaaddtoken{\textscg}%
 \else\ifx H#1\tipaaddtoken{\textsch}%
 \else\ifx J#1\tipaaddtoken{\textscj}%
 \else\ifx L#1\tipaaddtoken{\textscl}%
 \else\ifx N#1\tipaaddtoken{\textscn}%
 \else\ifx Q#1\tipaaddtoken{\TIPAtextscq}%
 \else\ifx R#1\tipaaddtoken{\textscr}%
 \else\ifx U#1\tipaaddtoken{\textscu}%
 \else\tipaaddtoken{#1}%
 \fi\fi\fi\fi\fi\fi\fi\fi\fi\fi\fi
 \tipatestforonechar
}
\def\gettipacolon\:#1{%
 \ifx d#1\tipaaddtoken{\textrtaild}%
 \else\ifx l#1\tipaaddtoken{\textrtaill}%
 \else\ifx n#1\tipaaddtoken{\textrtailn}%
 \else\ifx r#1\tipaaddtoken{\textrtailr}%
 \else\ifx R#1\tipaaddtoken{\textturnrrtail}%
 \else\ifx s#1\tipaaddtoken{\textrtails}%
 \else\ifx t#1\tipaaddtoken{\textrtailt}%
 \else\ifx z#1\tipaaddtoken{\textrtailz}%
 \else\tipaaddtoken{#1}%
 \fi\fi\fi\fi\fi\fi\fi\fi
 \tipatestforonechar
}
\def\gettipaexclam\!#1{%
 \ifx b#1\tipaaddtoken{\texthtb}%
 \else\ifx d#1\tipaaddtoken{\texthtd}%
 \else\ifx g#1\tipaaddtoken{\texthtg}%
 \else\ifx j#1\tipaaddtoken{\texthtbardotlessj}%
 \else\ifx G#1\tipaaddtoken{\texthtscg}%
 \else\ifx o#1\tipaaddtoken{\textbullseye}%
 \else\tipaaddtoken{#1}%
 \fi\fi\fi\fi\fi\fi
 \tipatestforonechar
}
\gdef\gettipavert\|#1{%
 \ifx [#1\tipaaddtoken{\tipasubbridgeaccent}%
 \else\ifx ]#1\tipaaddtoken{\tipainvsubbridgeaccent}%
 \else\ifx (#1\tipaaddtoken{\tipasublhalfringaccent}%
 \else\ifx )#1\tipaaddtoken{\tipasubrhalfringaccent}%
 \else\ifx c#1\tipaaddtoken{\tiparoundcapaccent}%
 \else\ifx +#1\tipaaddtoken{\tipasubplusaccent}%
 \else\ifx '#1\tipaaddtoken{\tiparaisingaccent}%
 \else\ifx `#1\tipaaddtoken{\tipaloweringaccent}%
 \else\ifx <#1\tipaaddtoken{\tipaadvancingaccent}%
 \else\ifx >#1\tipaaddtoken{\tiparetractingaccent}%
 \else\ifx x#1\tipaaddtoken{\tipaovercrossaccent}%
 \else\ifx w#1\tipaaddtoken{\tipasubwaccent}%
 \else\ifx m#1\tipaaddtoken{\tipaseagullaccent}%
 \else\tipaaddtoken{#1}%
 \fi\fi\fi\fi\fi\fi\fi\fi\fi\fi\fi\fi\fi
 \tipatestforonechar
}

% With parsing, \activatetipa doesn't need to do as much.
\def\activatetipa{%
 %
}

\def\ttipatestforonechar#1#2!@!{%
 \ifx\relax#2\relax
  \def\next{\implementTIPAtext#1\endgroup}%
 \else
   \def\next{\tipacatchg{#1#2}\endgroup}%
 \fi \next
}
\def\parseTIPAtext#1{%
 \implementTIPAtext#1%
}

% default for these \TIPA… macros  is to just return #1
% But \* first makes it into a non-active character.
%
\def\DeclareTIPAstarmacro#1#2{%
 \expandafter\def\csname#1\string#2\endcsname##1{%
  \expandafter\@text@composite\csname#1\string#2\endcsname##1\@empty\@text@composite
  {{\endlinechar=-1 \unsetTIPAcatcodes\scantokens{##1}}}}%
}
\def\DeclareTIPAmacro#1#2{%
 \expandafter\def\csname#1\string#2\endcsname##1{%
  \expandafter\@text@composite\csname#1\string#2\endcsname##1\@empty\@text@composite
  {##1}}%
}

\DeclareRobustCommand{\T}[1]{\~{\m{#1}}}
\let\tipasafemode\relax

%: separator

{
 \gdef\real@five{5}%
 \gdef\real@four{4}%
 \gdef\real@three{3}%
 \gdef\real@two{2}%
 \gdef\real@one{1}%
 \catcode`5\active
 \catcode`4\active
 \catcode`3\active
 \catcode`2\active
 \catcode`1\active
 \gdef\TIPAresetdigits{%
  \edef1{\real@one}%
  \edef2{\real@two}%
  \edef3{\real@three}%
  \edef4{\real@four}%
  \edef5{\real@five}%
 }
}
\def\TIPAtonebar#1{\TIPAtonebari#1!!@!!}
\def\TIPAtonebari#1#2!!@!!{\TIPAtonebarx{#1}%
 \def\nextchar{#2}\ifx\nextchar\@empty
 \else
  \TIPAtonebar{#2}%
 \fi
}
\def\TIPAtonebarx#1{{\TIPAresetdigits
 \edef\TIPAtonedata{#1}\expandafter\tonebar
 \expandafter{\TIPAtonedata}}}

\def\TIPAstonebar#1{\TIPAstonebari#1!!@!!}
\def\TIPAstonebari#1#2!!@!!{\TIPAtonebarx{#1}%
 \def\nextchar{#2}\def\firstchar{#2}\ifx\nextchar\@empty
 \else
  \ifx\nextchar\firstchar
  \else
  \TIPAstonebar{#2}%
 \fi\fi
}

\def\TIPArtonebar#1{\TIPArtonebari#1!!@!!}
\def\TIPArtonebari#1#2!!@!!{\TIPArtonebarx{#1}%
 \def\nextchar{#2}\def\firstchar{#2}\ifx\nextchar\@empty
 \else
  \ifx\nextchar\firstchar
  \else
  \TIPArtonebar{#2}%
 \fi\fi
}
\def\TIPArtonebarx#1{{\TIPAresetdigits
 \edef\TIPAtonedata{#1}\expandafter\rtonebar
 \expandafter{\TIPAtonedata}}}

% faking right tone bars from left ones
\RequirePackage{graphicx}
\def\TIPAfakertonebar#1{{\TIPAreversedata#1!!@!!%
 \reflectbox{\expandafter\TIPAfakertonebari\expandafter{\TIPAtonedata}}}}
\def\TIPAfakertonebari#1{\TIPAfakertonebarii#1!!@!!}
\def\TIPAfakertonebarii#1#2!!@!!{\TIPAtonebarx{#1}%
 \def\nextchar{#2}\ifx\nextchar\@empty
 \else
  \TIPAfakertonebari{#2}%
 \fi
}
\def\TIPAreversedata#1#2!!@!!{%
 \expandafter\def\expandafter\TIPAtonedata\expandafter{\expandafter#1\TIPAtonedata}%
 \ifx\relax#2\relax\let\next\relax
 \else
  \def\next{\TIPAreversedata#2!!@!!}%
 \fi \next }
\def\TIPAtonedata{}
\def\UseFakeRightTones{\RequirePackage{graphicx}%
 \let\TIPArtonebar\TIPAfakertonebar}

\endinput
